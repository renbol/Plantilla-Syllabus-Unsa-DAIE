\documentclass[12pt]{article}
\usepackage[spanish]{babel}         % Utiliza los nombres de figuras, etc. En el idioma que se quiera
\usepackage[latin1]{inputenc} 
%\usepackage{natbib}
\usepackage{url}
%\usepackage[utf8x]{inputenc}
\usepackage{amsmath}
\usepackage{graphicx}
\graphicspath{{images/}}
\usepackage{parskip}
\usepackage{fancyhdr}

\usepackage{amssymb,amsthm,amsmath}
\usepackage{amsfonts}
\usepackage{multirow}
\usepackage{array}
\usepackage[export]{adjustbox}
\usepackage{lastpage}

\newcolumntype{x}[1]{%
>{\centering\hspace{0pt}}p{#1}}%

\usepackage{vmargin}
\setmarginsrb{1.5 cm}{1.0 cm}{1.5 cm}{1.0 cm}{1 cm}{1.0 cm}{1 cm}{1.0 cm}

\title{Laboratorio 1 - Formato A4 }			% Title
\author{Ing. Renzo Bol\'i�var Valdivia}								% Author
\date{ \today }											% Date

\clubpenalty=10000  %Control Lineas viudas
\widowpenalty=10000  %Control de lineas huerfanas

\usepackage[none]{hyphenat} % Para no Separa Palabras y "sloppy"  al inicio de documento  



\makeatletter
\let\thetitle\@title
\let\theauthor\@author
\let\thedate\@date
\makeatother

%\pagestyle{fancy}
%\fancyhf{}
%\rhead{\theauthor}
%\lhead{\thetitle}
%\cfoot{\thepage}

\usepackage{enumerate}
\usepackage{datetime}
\newdateformat{monthyeardate}{\THEYEAR}


\renewcommand*{\arraystretch}{1.35}
%\newenvironment{Tabular}[2][1]
%  {\def\arraystretch{#1}\tabular{#2}}
%  {\endtabular}


%%%%Modifica las Secciones
\renewcommand{\thesection}{\Roman{section}} %%ROMAN
\usepackage{titlesec}   %%%paquete especial
%%%%%%%Modifica la forma de las seccciones y tama�os
\titleformat{\section}[block]
  {\fontsize{12}{15}\bfseries\sffamily\filright}
  {\thesection}
  {1em}
  {\MakeUppercase}
%%%%%%%%%%%%%%%%


\begin{document}
\sloppy % para no separar palabras
%%%%%%%%%%%%%%%%%%%%%%%%%%%%%%%%%%%%%%%%%%%%%%%%%%%%%%%%%%%%%%%%%%%%%%%%%%%%%%%%%%%%%%%%%



%%%%%%%%%%%%%%%%%%%%%%%%%%%%%%%%%%%%%%%%%%%%%%%%%%%%%%%%%%%%%%%%%%%%%%%%%%%%%%%%%%%%%%%%%

%\tableofcontents
%\pagebreak

%%%%%%%%%%%%%%%%%%%%%%%%%%%%%%%%%%%%%%%%%%%%%%%%%%%%%%%%%%%%%%%%%%%%%%%%%%%%%%%%%%%%%%%%%

\begin{center}
\begin{figure}

\begin{tabular}[t]{c x{15cm} r}
  \centering
    \multirow{4}{*}{\includegraphics[width=0.20\textwidth]{img/unsa.png}} & { \small \bf  UNIVERSIDAD NACIONAL DE SAN AGUST�N DE AREQUIPA}  &  \\
                                                                       &  { \small \bf VICE RECTORADO ACAD�MICO}   & \\
                                                                       &  {\small  \bf FACULTAD DE INGENIER�A DE PRODUCCI�N Y SERVICIOS}  &     \\
                                                                       &  {\footnotesize  \bf DEPARTAMENTO ACAD�MICO DE INGENIER�A ELECTR�NICA}  &     \\ \\
                                                                       &  {\small \bf SILABO 2017 B}     &  \\ \\
                                                                       &  {\small \bf CURSO: COMPUTACI�N I}  &     \\
  \end{tabular}
\end{figure}


\end{center}





\section{DATOS GENERALES}

\begin{table}[h] \small
\begin{tabular}[10]{|p{0.27\textwidth}|p{0.19\textwidth}|p{0.23\textwidth} p{0.20\textwidth}|}
\hline
\bf Periodo Acad�mico : & \multicolumn{3}{|l|}{ 2017-A}      \\ \hline
\bf Escuela Profesional :    & \multicolumn{2}{|l}{ Ingenier�a en Telecomunicaciones}     & \\ \hline
\bf C�digo de Asignatura :    & \multicolumn{3}{|l|}{ 1702110}      \\ \hline
\bf Nombre de Asignatura :    & \multicolumn{3}{|l|}{ Computaci�n I}      \\ \hline
\bf Semestre :    & \multicolumn{3}{|l|}{ I (Primero)}      \\ \hline
\bf Caracteristicas :    & \multicolumn{3}{|l|}{ Semestral}      \\ \hline
\bf Duraci�n :    & \multicolumn{3}{|l|}{ 17 Semanas}      \\ \hline
\bf \multirow{4}{4cm} {Numero de Horas: (Semestral) } & Te�ricas (Lab.) : & 3 &    \\ \cline{2-4}
																		& Laboratorio : & 2 &    \\  \cline{2-4}
																		& Te�rico - Practicas : & 3 &    \\  \cline{2-4}
																		& Practicas : & 3 &    \\ \hline
\bf Numero de Cr�ditos :    & \multicolumn{3}{|l|}{ 3}      \\ \hline
\bf Prerequisitos :    & 1701103 &   Razonamiento L�gico &  \\ \hline

\end{tabular}
\end{table}


\section{DATOS ADMINISTRATIVOS}

%\begin{table}[h] \small
\scalebox{0.9}{
\begin{tabular}[10]{|p{0.20\textwidth}|p{0.17\textwidth}|p{0.13\textwidth}|p{0.17\textwidth}|p{0.17\textwidth}|p{0.10\textwidth}|}
\hline
 \multicolumn{6}{|l|}{ \bf{PROFESOR:} Renzo Gustavo Bol�var Valdivia  }      \\ 
 \multicolumn{6}{|l|}{ \bf{GRADO ACAD�MICO:} Bachiller en Ingenier�a Electr�nica}      \\ 
 \multicolumn{6}{|l|}{ \bf Candidato para Magister Ingenier�a de Proyectos   }      \\ 
 \multicolumn{6}{|l|}{ \bf{DEPARTAMENTO ACAD�MICO:} Ingenier�a Electr�nica  }      \\ \hline
 \bf \multirow{3}{3.6cm} {\bf HORARIO: Total Semanal 15 Hrs. 3 Grupos } & Lunes  & Martes & Miercoles & Jueves & Viernes    \\ \cline{2-6}
																		                                   & 8:40 - 11:10(A)  &  & 8:40 - 11:10(B) & 8:40 - 11:10(C) &     \\ \cline{2-6}
																										& 11:10 - 12:50(B)  &  & 11:10 - 12:50(C)  & 11:10 - 12:50(A)  &     \\ \hline
\bf AULA & Laboratorio  &  & Laboratorio & Laboratorio &     \\ \hline
																		




\end{tabular}}



\section{FUNDAMENTACI�N (JUSTIFICACI�N)}



\section{COMPETENCIAS DEL CURSO}



\section{CONTENIDO TEMATICO POR COMPETENCIAS}





\section{ESTRATEGIAS DE ENSE�ANZA}


\section{EVALUACI�N}










%\end{table}

% \cite{bibtex}. The Table of Contents will be updated automatically.


%\hspace{1 cm}--- Linus


\nocite{*} % basico para cuando no citas la bibliografia
\bibliographystyle{ieeetr}
\bibliography{biblist}

\end{document}
